\documentclass[a4paper,12pt]{report} % добавить leqno в [] для нумерации слева

%%% Работа с русским языком
\usepackage{cmap}					% поиск в PDF
\usepackage{mathtext} 				% русские буквы в формулах
\usepackage[T2A]{fontenc}			% кодировка
\usepackage[utf8]{inputenc}			% кодировка исходного текста
\usepackage[english,russian]{babel}	% локализация и переносы

\usepackage{graphicx}				%вставка изображений(графиков, в частности)
\usepackage{alltt}

%%% Дополнительная работа с математикой
\usepackage{amsmath,amsfonts,amssymb,amsthm,mathtools} % AMS
\usepackage{icomma} % "Умная" запятая: $0,2$ --- число, $0, 2$ --- перечисление

%% Номера формул
\mathtoolsset{showonlyrefs=true} % Показывать номера только у тех формул, на которые есть \eqref{} в тексте.

%% Шрифты
\usepackage{euscript}	 % Шрифт Евклид
\usepackage{mathrsfs} % Красивый матшрифт

%% Свои команды
\DeclareMathOperator{\sgn}{\mathop{sgn}}

%\setlength\parindent{0ex}
%\setlength\parskip{0.3cm}

\renewcommand{\arraystretch}{1.8}

%%% Заголовок
\author{Волков Павел А-14-19}
\title{Отчет по Лабораторной работе №3}
\date{\today}

\begin{document}

\begin{titlepage}
	\newpage

	\begin{center}
	НАЦИОНАЛЬНЫЙ ИССЛЕДОВАТЕЛЬСКИЙ УНИВЕРСИТЕТ\\
		"МОСКОВСКИЙ ЭНЕРГЕТИЧЕСКИЙ ИНСТИТУТ"\\
	\end{center}

	\vspace{8em}	

	\begin{center}
		\Large Кафедра математического и компьютерного моделирования\\ 
	\end{center}

	\vspace{2em}

	\begin{center}
		\textsc{\textbf{ \Large Численные методы \linebreak Отчет по лабораторной работе №3 \linebreak "Решение систем линейных алгебраических уравнений прямыми методами. Теория возмущений." \linebreak Вариант 33}}
	\end{center}

	\vspace{6em}



	\newbox{\lbox}
	\savebox{\lbox}{\hbox{Амосова Ольга Алексеевна}}
	\newlength{\maxl}
	\setlength{\maxl}{\wd\lbox}
	\hfill\parbox{11cm}{
		\hspace*{5cm}\hspace*{-5cm}Студент:\hfill\hbox to\maxl{Волков Павел Евгеньевич\hfill}\\
		\hspace*{5cm}\hspace*{-5cm}Преподаватель:\hfill\hbox to\maxl{Амосова Ольга Алексеевна}\\
		\\
		\hspace*{5cm}\hspace*{-5cm}Группа:\hfill\hbox to\maxl{А-14-19}\\
	}


	\vspace{\fill}

	\begin{center}
		Москва \\2021
	\end{center}

\end{titlepage}

\section*{Задача 3.1}
\subsection*{Постановка задачи}

Реализовать решение СЛАУ с помощью $LU$-разложения и $LU$-разложения по схеме частичного выбора.
Решить систему небольшой размерности с возмущенной матрицей обоими методами, оценить погрешность и сравнить с теоретической оценкой. Проанализировать поведение метода с ростом числа уравнений.

$(33 + 3) mod 2 = 0$ - решение с помощью $LU$-разложения  реализовано в виде 2-х функций, одна из которых возвращает две матрицы - $L$ и $U$, не модифицируя $A$, а вторая функция решает систему, решение с помощью $LU$ по схеме частичного выбора модифицирует исходную матрицу $A$.

\[
	(33+3) mod 4 = 0 \to A_{i, j} = \tg^{17-j}(i + 1)
\]
\subsection*{Решение}

\section*{Задача 3.2}
\subsection*{Постановка задачи}

Дана система уравнений $Ax = b$ порядка $n$ с разреженной матрицей $A$. Решить систему прямым методом.

В случае коллизий в матрице, диагонали имеют приоритет над столбцами, главные диагонали - над побочными.

$n = 65$ На главной диагонали элементы равны 87, на 23-ей наддиагонали элементы равны 30, на 2-ой побочной поддиагонали элементы равны 4. ($b_i = n \cdot i + n$) 
\subsection*{Решение}

\section*{Задача 3.3}
\subsection*{Постановка задачи}

Решить задачу итерационным методом, указанным в индивидуальном варианте. Вектор правой части задается как $b = Ax$, где $x_i = 33$

Элементы матрицы $A$ задаются формулами \newline $a_{i, j} = \dfrac{\cos{i + j}}{0.1 \cdot \beta} + 0.1\beta \cdot e^{-(i - j)^2}$.

Параметр $\beta$ задается формулой $\beta = (|66 - 33| + 5) \cdot m$, здесь N - номер варианта, m - размерность матрицы, указанная в варианте. Вектор b задается по вектору решения.

m = 26, метод минимальных невязок.


\subsection*{Решение}

\end{document}