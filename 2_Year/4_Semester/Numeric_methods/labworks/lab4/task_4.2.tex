\section*{Задача 4.2}
\subsection*{Постановка задачи}

Дана функция $y = f(x)$. Приблизить функцию методом интерполяции, используя многочлен Лагранжа. Степень $n$ подобрать таким образом, чтобы максимальная величина погрешности на отрезке $[a, b]$ не превышала заданной величины $\varepsilon$. Построить графики многочленов и графики погрешностей. Приблизить функцию методом интерполяции, указанным в индивидуальном варианте (Квадратичный сплайн с дополнительным условием $y'(a) = f'(a)$). Сравнить полученные результаты.
\begin{gather}
	f(x) = x\sin{(2 - x)}, [1, 4] \\
	\varepsilon = 0.001
\end{gather}
	
\subsection*{Теоретический материал}



Так как имеется свобода выбора отрезков разбиения, то имеет смысл выбрать точки таким образом, чтобы погрешность интерполяции была минимальной. Для этого воспользуемся свойством наименее уклонятся от нуля на отрезке $[-1, 1]$ многочленов Чебышева. То есть в качестве узлов интерполяции возьмем нули многочлена Чебышева:

\[
	x_k = \dfrac{a + b}{2} + \dfrac{b - a}{2}\cos{\left(\pi\dfrac{2k + 1}{2n + 2}\right)}, k = 0, 1\dots, n
\]

\textbf{Определение}{
	Сплайном степени $m$ называется функция $S_m(x)$, обладающая следующими свойствами:
	\begin{enumerate}
		\item Функция $S_m(x)$ непрерывна на $[a, b]$ со своими производными.
		\item На каждом отрезке $[x_i, x_{i+1}]$ функция $S_m(x)$ совпадает с некоторым алгебраическим многочленом $P_{m, i}(x)$ степени $m$.
		\item $S_m(x_i) = y_i$, $i = 0, 1\dots n$
	\end{enumerate}
}

\textbf{Определение}{
	Величина $R_n(x) = |f(x) - P_n(x)|$ называется остаточным членом
интерполяции или погрешностью интерполяции.
}

\textbf{Теорема}{
	Пусть функция $f(x)$ дифференцируема $(n+1)$ раз на отрезке $[a, b]$, содержащем узлы интерполяции $x_i$. Тогда для погрешности интерполяции в точке $x \in [a, b]$ справедлива оценка:
	\[
		R_n(x) = |f(x) - P_n(x)| \leq \dfrac{M_{n+1}}{(n+1)!}|\omega_{n+1}(x)|
	\]
	Где $M_{n+1} = max|f^{(n+1)}(x)|$, а $\omega_{n+1}(x) = (x - x_0)\dots(x - x_n)$
}

\subsection*{Решение}

Коэффициенты многочленов сплайна будем искать следующим образом:

\begin{enumerate}
	\item Запишем многочлены в форме $P_i(x) = a_i + b_i(x - x_i) + c_i(x - x_i)(x - x_{i+1})$.
	\item Из условий интерполяции получаем: $a_i = f(x_i)$ а также $a_i + b_i(x_{i+1} - x_i) = a_{i+1} = f(x_{i+1})$, то есть $b_i = \dfrac{a_{i+1} - a_i}{x_{i+1} - x_i}$.
	\item Коэффициенты $c_i$ будем определять из условий непрерывности производной: $P_i'(x_{i+1}) = P_{i+1}'(x_{i+1})$. То есть: $b_i - 2c_ix_i = b_{i+1} - 2x_{i+2}c_{i+1}$. В конечном итоге получаем следующую формулу для $c_{i+1}$: $c_{i+1} = \dfrac{b_{i+1} - b_i}{2x_{i+2}} + \dfrac{x_i}{x_{i+2}}c_i$.
\end{enumerate}

\subsection*{Анализ результатов}
\subsection*{Код программы}
