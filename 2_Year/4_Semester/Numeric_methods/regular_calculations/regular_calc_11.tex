\documentclass[a4paper,12pt]{report} % добавить leqno в [] для нумерации слева

%%% Работа с русским языком
\usepackage{cmap}					% поиск в PDF
\usepackage{mathtext} 				% русские буквы в формулах
\usepackage[T2A]{fontenc}			% кодировка
\usepackage[utf8]{inputenc}			% кодировка исходного текста
\usepackage[english,russian]{babel}	% локализация и переносы

%%% Дополнительная работа с математикой
\usepackage{amsmath,amsfonts,amssymb,amsthm,mathtools} % AMS
\usepackage{icomma} % "Умная" запятая: $0,2$ --- число, $0, 2$ --- перечисление

%% Номера формул
\mathtoolsset{showonlyrefs=true} % Показывать номера только у тех формул, на которые есть \eqref{} в тексте.

%% Шрифты
\usepackage{euscript}	 % Шрифт Евклид
\usepackage{mathrsfs} % Красивый матшрифт

%% Свои команды
\DeclareMathOperator{\sgn}{\mathop{sgn}}

%\setlength\parindent{0ex}
%\setlength\parskip{0.3cm}

%%% Заголовок
\author{Волков Павел А-14-19}
\title{Типовой расчет №11 по численным методам Вариант 3}
\date{\today}

\usepackage{graphicx}

\begin{document} % конец преамбулы, начало документа

\maketitle

\newpage
\section*{Задание}
Дана система уравнений $Ax = b$. Привести ее к виду, удобному для итераций, проверить выполнение достаточного условия сходимости указанных ниже методов. Выполнить 3 итерации по методу Якоби и 3 итерации по методу Зейделя. Определить, во сколько раз уменьшится норма невязки в каждом случае. Используя, апостериорную оценку, вычислить погрешность приближенного решения, полученного на 3 итерации каждого метода.

УКАЗАНИЕ. Для обеспечения выполнения достаточного  условия сходимости воспользоваться перестановкой строк в исходной системе уравнений.

\begin{gather*}
	A = 
	\begin{pmatrix}
		-1 & 9 & -8 &  121 \\
		-4 & 0 & 29 & -1 \\
		-8 & 96 & -4 & 1 \\
		91 & 5 & -7 & -5
	\end{pmatrix},
	b = 
	\begin{pmatrix}
		-110 \\ 220 \\ 675 \\ -373
	\end{pmatrix}
\end{gather*}
\section*{Решение}

Для выполнения достаточного условия сходимости должно иметь место диагональное преобладание элементов матрицы. Переставим строки соответствующим образом и запишем расчетные формулы для методов Якоби и Зейделя в матричном виде:
\begin{align*}
	A &= 
	\begin{pmatrix}
		91 & 5 & -7 & -5 \\
		-8 & 96 & -4 & 1 \\
		-4 & 0 & 29 & -1 \\
		-1 & 9 & -8 &  121
	\end{pmatrix}
& 
	B &= 
	\begin{pmatrix}
		0 & -\frac{5}{91} & \frac{1}{13} & \frac{5}{91} \\
		\frac{1}{12} & 0 & \frac{1}{24} & -\frac{1}{96} \\
		\frac{4}{29} & 0 & 0 & \frac{1}{29} \\
		\frac{1}{121} & -\frac{9}{121} & \frac{8}{121} & 0 \\
	\end{pmatrix} \\
\\
	B_1 &= 
	\begin{pmatrix}
		0 & 0 & 0 & 0 \\
		\frac{1}{12} & 0 & 0 & 0 \\
		\frac{4}{29} & 0 & 0 & 0 \\
		\frac{1}{121} & -\frac{9}{121} & \frac{8}{121} & 0 \\
	\end{pmatrix}
&
	B_2 &= 
	\begin{pmatrix}
		0 & -\frac{5}{91} & \frac{1}{13} & \frac{5}{91} \\
		0 & 0 & \frac{1}{24} & -\frac{1}{96} \\
		0 & 0 & 0 & \frac{1}{29} \\
		0 & 0 & 0 & 0 \\
	\end{pmatrix}
\end{align*}

Метод Якоби(Метод простых итераций): $x^{(n+1)} = Bx^{(n)} + b^*$

Метод Зейделя: $x^{(n+1)} = B_1x^{(n+1)} + B_2x^{(n)} + b^*$ или:
$A = A_1 + D + A_2$, где $A_1$ - нижнетреугольная матрица, $D$ - диагональная матрица, $A_2$ - верхнетреугольная.
$(A_1 + D)x = b - A_2x$. Пусть $A_1 + D = L'$ Тогда расчетную формулу метода Зейделя можно записать в следующем виде: 
$x^{(n+1)} = L'^{-1}(b - A_2x^{(n)})$

Выполним 3 итерации метода Якоби(в качестве начального приближения возьмем вектор $b^*$):
\begin{gather*}
	0)x^{(0)} = (-1.209,  2.292, 23.276, -3.083)^T, r^{(0)} = (-136.1,  -86.5, 7.9, -164.4)^T, ||r^{(0)}||_e = 230.387\\
	1)x^{(1)} = (0.286,  3.193, 23.003, -1.724)^T, r^{(1)} = (-0.375, -9.511, -7.339,  8.800)^T, ||r^{(1)}||_e = 14.896\\
	2)x^{(2)} =  (0.290, 3.292, 23.256, -1.797)^T, r^{(2)} =  (-0.913, -1.118,  0.056, -1.137)^T, ||r^{(2)}||_e =  1.838\\
	3)x^{(3)} =  (0.301, 3.304, 23.254, -1.788)^T, r^{(3)} =  (0.025, -0.063, -0.050, 0.110)^T, ||r^{(3)}||_e =  0.139
\end{gather*}
Таким образом, норма невязки уменьшилась в $\dfrac{||r^{(0)}||}{||r^{(3)}||} = 230.387/0.139 = 1657$ раз.
Апостериорная оценка метода Якоби: \newline $||x^{(3)} - \bar{x}|| < \dfrac{||B||}{1 - ||B||} \cdot ||x^{(3)} - x^{(2)}|| =0.005281 $, где $||B|| = 0.225677$.

Выполним 3 итерации метода Зейделя(в качестве начального приближения возьмем вектор $b$):
\begin{gather*}
	0)x^{(0)} = (-1.209,  2.292, 23.276, -3.083)^T,\\
		r^{(0)} = (-136.1,  -86.5, 7.9, -164.4)^T, ||r^{(0)}||_e = 230.387\\
	1)x^{(1)} = (0.286,  3.317, 23.209,   -1.792)^T,\\
		r^{(1)} = (-0.853,   1.557, -1.290,  0)^T, ||r^{(1)}||_e =2.195\\
	2)x^{(2)} =  (0.295,  3.302, 23.254, -1.788)^T, \\
		r^{(2)} =  (-0.418, -0.178, -0.004,  0)^T, ||r^{(2)}||_e = 0.456\\
	3)x^{(3)} =  (0.301,  3.304, 23.256, -1.788)^T, \\
		r^{(3)} =  (0,006, 0.003, 0.00007,  0)^T, ||r^{(3)}||_e = 0.007
\end{gather*}
Таким образом, норма невязки уменьшилась в $\dfrac{||r^{(0)}||}{||r^{(3)}||} = 230.387/0.007 = 32912$ раз.
Апостериорная оценка метода Зейделя $||x^{(3)} - \bar{x}|| < \dfrac{||B_2||}{1 - ||B||} \cdot ||x^{(3)} - x^{(2)}|| = 
 \dfrac{0.122429}{1 - 0.225677} \cdot 0.005182 = 0.000819$
\end{document}