\documentclass[a4paper,12pt]{report} % добавить leqno в [] для нумерации слева

%%% Работа с русским языком
\usepackage{cmap}					% поиск в PDF
\usepackage{mathtext} 				% русские буквы в формулах
\usepackage[T2A]{fontenc}			% кодировка
\usepackage[utf8]{inputenc}			% кодировка исходного текста
\usepackage[english,russian]{babel}	% локализация и переносы

%%% Дополнительная работа с математикой
\usepackage{amsmath,amsfonts,amssymb,amsthm,mathtools} % AMS
\usepackage{icomma} % "Умная" запятая: $0,2$ --- число, $0, 2$ --- перечисление

%% Номера формул
\mathtoolsset{showonlyrefs=true} % Показывать номера только у тех формул, на которые есть \eqref{} в тексте.

%% Шрифты
\usepackage{euscript}	 % Шрифт Евклид
\usepackage{mathrsfs} % Красивый матшрифт

%% Свои команды
\DeclareMathOperator{\sgn}{\mathop{sgn}}

%\setlength\parindent{0ex}
%\setlength\parskip{0.3cm}

%%% Заголовок
\author{Волков Павел А-14-19}
\title{Типовой расчет №15 по численным методам Вариант 3}
\date{\today}

\usepackage{graphicx}

\begin{document} % конец преамбулы, начало документа

\maketitle

\newpage
\section*{Задание}

Для функции $y = y(x)$, заданной таблицей своих значений, построить интерполяционные многочлены в форме Лагранжа и Ньютона. Используя их, вычислить приближеное значение функции в точке \~ x = -0.24.

\[
\begin{tabular}{ | c | c | c | c | c |}
	\hline
	$x$ & -2 & -1 & 0 & 1  \\ \hline
	$y$ & -2 & 0 & 1 & -3  \\ \hline
\end{tabular}
\]

\section*{Решение}

Запишем интерполяционный многочлен в форме Лагранжа:
\begin{multline}
	L_3(x) = -2\dfrac{(x + 1)x(x - 1)}{(-2 + 1)(-2 - 0)(-2 - 1)} + \dfrac{(x + 2)(x + 1)(x - 1)}{(0 + 2)(0 + 1)(0 - 1)} 
	- 3\dfrac{(x + 2)(x + 1)x}{(1 + 2)(1 + 1)(1 - 0)} = \\ = \dfrac{1}{3}(x + 1) x (x - 1) - \dfrac{1}{2}(x + 2)(x + 1)(x - 1)
	- \dfrac{1}{2}(x + 2)(x+1)x = \\ = \dfrac{1}{3}(x + 1) x (x - 1) - \dfrac{1}{2}(x + 2)(x + 1)(2x - 1)
\end{multline}

Построим интерполяционный многочлен в форме Ньютона с конечными разностями:

$
\begin{array}{| c | c | c | c | c |}
	\hline
	& f(x_0) & \Delta f_0 & \Delta^2 f_0 & \Delta^3 f_0  \\ \hline
	-2 & -2 &&& \\ \hline
	&&2&& \\ \hline
	-1 & 0 &&-1& \\ \hline
	&&1&&-4 \\ \hline
	0 & 1 &&-5& \\ \hline
	&&-4&& \\ \hline
	1 & -3 &&&\\ \hline
\end{array}
$

\begin{multline}
	P_3(x) = -2 + 2(x + 2) - \dfrac{1}{2}(x + 2)(x + 1) - \dfrac{4}{6}(x + 2)(x + 1)x
\end{multline}

Таким образом, приближенное значение функции $f(x)$ в точке  \~ x = -0.24 составляет: 1.065216

\end{document}