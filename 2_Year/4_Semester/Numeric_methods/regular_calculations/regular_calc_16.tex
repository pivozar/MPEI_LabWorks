\documentclass[a4paper,12pt]{report} % добавить leqno в [] для нумерации слева

%%% Работа с русским языком
\usepackage{cmap}					% поиск в PDF
\usepackage{mathtext} 				% русские буквы в формулах
\usepackage[T2A]{fontenc}			% кодировка
\usepackage[utf8]{inputenc}			% кодировка исходного текста
\usepackage[english,russian]{babel}	% локализация и переносы

%%% Дополнительная работа с математикой
\usepackage{amsmath,amsfonts,amssymb,amsthm,mathtools} % AMS
\usepackage{icomma} % "Умная" запятая: $0,2$ --- число, $0, 2$ --- перечисление

%% Номера формул
\mathtoolsset{showonlyrefs=true} % Показывать номера только у тех формул, на которые есть \eqref{} в тексте.

%% Шрифты
\usepackage{euscript}	 % Шрифт Евклид
\usepackage{mathrsfs} % Красивый матшрифт

%% Свои команды
\DeclareMathOperator{\sgn}{\mathop{sgn}}

%\setlength\parindent{0ex}
%\setlength\parskip{0.3cm}

%%% Заголовок
\author{Волков Павел А-14-19}
\title{Типовой расчет №16 по численным методам Вариант 3}
\date{\today}

\usepackage{graphicx}

\begin{document} % конец преамбулы, начало документа

\maketitle

\newpage
\section*{Задание}

Функция $y = y(x)$, задана таблицей своих значений. Вычислить приближенное значение функции в точке \~ x = 0.92, используя интерполяционные многочлены Ньютона первой, второй и третьей степеней. Для каждого вычисленного значения найти практическую оценку погрешности. Записать все результаты с учетом погрешности.

УКАЗАНИЕ: Перед построением многочленов следует переупорядочить таблицу, расположив точки в порядке удаления от \~ x.

\[
\begin{array}{| c | c | c | c | c | c |}
	\hline
	x & 0.8 & 1.6 & 0 & 2.4 & 3.2  \\ \hline
	y & 2.5 & 4.8 & 1 & 8.3 & 13  \\ \hline
\end{array}
\]

\section*{Решение}

Составим таблицу разделенных разностей:
\[
\begin{array}{| c | c | c | c | c |}
	\hline
	     & f(x_0) & f(x_i, x_{i+1}) & f(x_i, x_{i+1}, x_{i+2}) & f(x_i, x_{i+1}, x_{i+2}, x_{i+3})  \\ \hline
	0.8 &    2.5  &                      &                                     & \\ \hline
	     &           &  2.875            &                                     & \\ \hline
	1.6 &   4.8   &                      &  0.3125                         & \\ \hline
	      &          &  2.375            &                                     &  -0.014467(592) \\ \hline
	0    &    1    &                       &  0.2(7)                          & \\ \hline
	      &          &  3.041(6)        &                                     &  4.07986(1) \\ \hline
	2.4 &   8.3  &                       &   3.541(6)                    &\\ \hline
	      &         &  5.875             &                                     & \\ \hline
	3.2 &   13   &                       &                                     &\\ \hline
\end{array}
\]

\begin{gather}
	P_1(x) = 2.5 + 2.875(x - 0.8) \\
	P_1(0.92) = 2.845 \\
	P_2(x) = 2.5 + 2.875(x - 0.8) + 0.3125(x - 0.8)(x-1.6) \\
	P_2(0.92) = 2.845 + 0.3125 \cdot 0.12 \cdot 0.68 = 2.8195 \\
	P_3(x) = 2.5 + 2.875(x - 0.8) + 0.3125(x - 0.8)(x-1.6) - 0.014467(x - 0.8)(x - 1.6)x \\
	P_3(0.92) = 2.8195 - 0.014467 \cdot 0.12 \cdot 0.68 \cdot 0.92 = 2.82058606
\end{gather}

Практические оценки погрешности:

\noindent
$
	R_1(0.92) = 0.0255 \\
	R_2(0.92) = 0.00108606
$

Получаем следующий результат:

\noindent
$
	f(0.92) = 2.845 \pm 0.0255 \\
	f(0.92) = 2.8195 \pm 0.0012
$

\end{document}