\documentclass[a4paper,12pt]{report} % добавить leqno в [] для нумерации слева

%%% Работа с русским языком
\usepackage{cmap}					% поиск в PDF
\usepackage{mathtext} 				% русские буквы в формулах
\usepackage[T2A]{fontenc}			% кодировка
\usepackage[utf8]{inputenc}			% кодировка исходного текста
\usepackage[english,russian]{babel}	% локализация и переносы

%%% Дополнительная работа с математикой
\usepackage{amsmath,amsfonts,amssymb,amsthm,mathtools} % AMS
\usepackage{icomma} % "Умная" запятая: $0,2$ --- число, $0, 2$ --- перечисление

%% Номера формул
\mathtoolsset{showonlyrefs=true} % Показывать номера только у тех формул, на которые есть \eqref{} в тексте.

%% Шрифты
\usepackage{euscript}	 % Шрифт Евклид
\usepackage{mathrsfs} % Красивый матшрифт

%% Свои команды
\DeclareMathOperator{\sgn}{\mathop{sgn}}

%\setlength\parindent{0ex}
%\setlength\parskip{0.3cm}

%%% Заголовок
\author{Волков Павел А-14-19}
\title{Типовой расчет №5 по численным методам Вариант 3}
\date{\today}

\usepackage{graphicx}

\begin{document} % конец преамбулы, начало документа

\maketitle

\newpage
\section*{Задание}
Решить систему уравнений $Ax = b$ методом Гаусса (схема единственного деления)
\[
	A = 
	\begin{pmatrix}
		-5 & 1 & 1 & 0 \\
		15 & -6 & 5 & 1 \\
		50 & -16 & 11 & 9 \\
		0 & 6 & 24 & 44
	\end{pmatrix}, b = 
	\begin {pmatrix}
		-4 \\ -16 \\ -2 \\ 98
	\end{pmatrix}
\]

\section*{Решение}

Прямой ход метода Гаусса:

\begin{gather*}
	A = 
	\begin{pmatrix}
		-5 & 1 & 1 & 0 & -4\\
		15 & -6 & 5 & 1 & -16\\
		50 & -16 & 11 & 9 & -2\\
		0 & 6 & 24 & 44 & 98
	\end{pmatrix} \sim 
	\begin{pmatrix}
		-5 & 1 & 1 & 0 & -4\\
		0 & -3 & 8 & 1 & -28\\
		0 & -6 & 21 & 9 & -42\\
		0 & 6 & 24 & 44 & 98
	\end{pmatrix} \sim
	\begin{pmatrix}
		-5 & 1 & 1 & 0 & -4\\
		0 & -3 & 8 & 1 & -28\\
		0 & 0 & 5 & 7 & 14\\
		0 & 0 & 40 & 46 & 42
	\end{pmatrix} \sim \\ \sim
	\begin{pmatrix}
		-5 & 1 & 1 & 0 & -4\\
		0 & -3 & 8 & 1 & -28\\
		0 & 0 & 5 & 7 & 14\\
		0 & 0 & 0 & -10 & -70
	\end{pmatrix}
\end{gather*}

\noindent Обратный ход:

\begin{gather*}
	A = 
	\begin{pmatrix}
		-5 & 1 & 1 & 0 & -4\\
		0 & -3 & 8 & 1 & -28\\
		0 & 0 & 5 & 0 & -35\\
		0 & 0 & 0 & 1 & 7
	\end{pmatrix} \sim
	\begin{pmatrix}
		-5 & 1 & 1 & 0 & -4\\
		0 & 1 & 0 & 0 & -7\\
		0 & 0 & 1 & 0 & -7\\
		0 & 0 & 0 & 1 & 7
	\end{pmatrix} \sim
	\begin{pmatrix}
		1 & 0 & 0 & 0 & -2\\
		0 & 1 & 0 & 0 & -7\\
		0 & 0 & 1 & 0 & -7\\
		0 & 0 & 0 & 1 & 7
	\end{pmatrix}
\end{gather*}

\noindent Проверка решения:

\[
	\left\{
		\begin{aligned}
			-5 \cdot (-2) + 1 \cdot (-7) + 1 \cdot (-7) + 0 \cdot 7&= -4  \\
			15 \cdot (-2) + (-6) \cdot (-7) + 5 \cdot (-7) + 1 \cdot 7 &= -16\\
			50 \cdot (-2) + (-16) \cdot (-7) + 11 \cdot (-7) + 9 \cdot 7 &= -2\\
			0 \cdot (-2) + 6 \cdot (-7) + 24 \cdot (-7) + 44 \cdot 7 &= 98\\
		\end{aligned}
	\right.
\]

Ответ: $(-2, -7, -7, 7)^T$



\end{document}