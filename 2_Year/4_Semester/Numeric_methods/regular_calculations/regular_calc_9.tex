\documentclass[a4paper,12pt]{report} % добавить leqno в [] для нумерации слева

%%% Работа с русским языком
\usepackage{cmap}					% поиск в PDF
\usepackage{mathtext} 				% русские буквы в формулах
\usepackage[T2A]{fontenc}			% кодировка
\usepackage[utf8]{inputenc}			% кодировка исходного текста
\usepackage[english,russian]{babel}	% локализация и переносы

%%% Дополнительная работа с математикой
\usepackage{amsmath,amsfonts,amssymb,amsthm,mathtools} % AMS
\usepackage{icomma} % "Умная" запятая: $0,2$ --- число, $0, 2$ --- перечисление

%% Номера формул
\mathtoolsset{showonlyrefs=true} % Показывать номера только у тех формул, на которые есть \eqref{} в тексте.

%% Шрифты
\usepackage{euscript}	 % Шрифт Евклид
\usepackage{mathrsfs} % Красивый матшрифт

%% Свои команды
\DeclareMathOperator{\sgn}{\mathop{sgn}}

%\setlength\parindent{0ex}
%\setlength\parskip{0.3cm}

%%% Заголовок
\author{Волков Павел А-14-19}
\title{Типовой расчет №9 по численным методам Вариант 3}
\date{\today}

\usepackage{graphicx}

\begin{document} % конец преамбулы, начало документа

\maketitle

\newpage
\section*{Задание}
Вычислить нормы $|| \cdot ||_1, || \cdot ||_e, || \cdot ||_{\infty}$ матрицы $A$ и нормы $|| \cdot ||_1, || \cdot ||_e, || \cdot ||_{\infty}$ вектора $b$.

Считая, что компоненты вектора $b$ получены в результате округления по дополнению, найти его относительную погрешность в каждой из трех указанных норм.
\[
	A = 
	\begin{pmatrix}
		-1.585 & -2.048 & -0.372 \\
		-2.782 & 1.048 & 0.166 \\
		-0.005 & -2.177 & 1.33 
	\end{pmatrix}, b = 
	\begin {pmatrix}
		0.5 \\ -0.88 \\ 3.7
	\end{pmatrix}
\]

\section*{Решение}
Вычислим нормы матрицы $A$:

1-я норма:  $||A||_1 = max(4.005, 3.996, 3.512) = 4.005$

Евклидова норма: $||A||_e = 4.713656$

Бесконечная норма: $||A||_{\infty} = max(4.372, 5.273, 1.868) = 5.273$

\hspace{2cm}

\noindent Вычислим норму вектора $b$:

1-я норма:  $||b||_1 = 5.08 $

Евклидова норма: $||b||_e = 3.8359353$

Бесконечная норма: $||b||_{\infty} = 3.7$

\hspace{2cm}

Вектор погрешности $\begin{pmatrix} 0.05 \\ 0.005 \\ 0.05 \end{pmatrix}$.
Тогда
\begin{align*}
	\Delta b^*_1 &= 0.105 & \delta b^*_1 &=0.020669\\
	\Delta b^*_e &= 0.070872 & \delta b^*_e &= 0.018476\\
	\Delta b^*_{\infty} &= 0.05 & \delta b^*_{\infty} &= 0.009482
\end{align*}
\end{document}