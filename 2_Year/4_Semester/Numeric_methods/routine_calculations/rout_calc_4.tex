\documentclass[a4paper,12pt]{report} % добавить leqno в [] для нумерации слева

%%% Работа с русским языком
\usepackage{cmap}					% поиск в PDF
\usepackage{mathtext} 				% русские буквы в формулах
\usepackage[T2A]{fontenc}			% кодировка
\usepackage[utf8]{inputenc}			% кодировка исходного текста
\usepackage[english,russian]{babel}	% локализация и переносы

%%% Дополнительная работа с математикой
\usepackage{amsmath,amsfonts,amssymb,amsthm,mathtools} % AMS
\usepackage{icomma} % "Умная" запятая: $0,2$ --- число, $0, 2$ --- перечисление

%% Номера формул
\mathtoolsset{showonlyrefs=true} % Показывать номера только у тех формул, на которые есть \eqref{} в тексте.

%% Шрифты
\usepackage{euscript}	 % Шрифт Евклид
\usepackage{mathrsfs} % Красивый матшрифт

%% Свои команды
\DeclareMathOperator{\sgn}{\mathop{sgn}}

%\setlength\parindent{0ex}
%\setlength\parskip{0.3cm}

%%% Заголовок
\author{Волков Павел А-14-19}
\title{Типовой расчет №4 по численным методам Вариант 3}
\date{\today}

\usepackage{graphicx}

\begin{document} % конец преамбулы, начало документа

\maketitle

\newpage
\section*{Задание}
Найти корень нелинейного уравнения $f(x) = 0$, локализованный на отрезке $[a, b]$, методом Ньютона с точностью $\varepsilon = 10^{-8}$
\[
    f(x) = 2x + \frac{1}{\sqrt{x - 1}} - 6, [2, 5]
\]

\section*{Решение}

\subsection*{Вычисление производной}
Вычислим производную $f'$ функции $f(x) = 2x + \frac{1}{\sqrt{x - 1}}$

\[
    f'(x) = 2 - \frac{1}{2\sqrt{(x - 1)^3}}
\]

\noindent и запишем рекуррентное соотношение для метода Ньютона:
\[
    x^{(k + 1)} = x^{(k)} - \frac{f(x^{(k)})}{f'(x^{(k)})}
\]

\subsection*{Применение метода}
В качестве начального приближения возьмем середину отрезка локализации, то есть $x^{(0)} = 3.5$

\begin{tabular}{|| c | c | c ||}
    \hline
    $k$ & $x^{(k)}$ & $|x^{(k+1)} - x^{(k)}|$ \\ \hline
    0 & 3.5 & - \\ \hline
    1 & 2.628664113836 & 0.871335886163 \\ \hline
    2 & 2.605412493250 & 0.023251620586 \\ \hline
    3 & 2.605377940557 & 0.000034552692 \\ \hline
    4 & 2.605377940479 & 0.000000000078 \\ \hline
\end{tabular}

Ответ: $\overline{x} = x^{(4)} \pm \varepsilon = 2.60537794 \pm 0.00000001$

\end{document}