\documentclass[a4paper,12pt]{report} % добавить leqno в [] для нумерации слева

%%% Работа с русским языком
\usepackage{cmap}					% поиск в PDF
\usepackage{mathtext} 				% русские буквы в формулах
\usepackage[T2A]{fontenc}			% кодировка
\usepackage[utf8]{inputenc}			% кодировка исходного текста
\usepackage[english,russian]{babel}	% локализация и переносы

%%% Дополнительная работа с математикой
\usepackage{amsmath,amsfonts,amssymb,amsthm,mathtools} % AMS
\usepackage{icomma} % "Умная" запятая: $0,2$ --- число, $0, 2$ --- перечисление

%% Номера формул
\mathtoolsset{showonlyrefs=true} % Показывать номера только у тех формул, на которые есть \eqref{} в тексте.

%% Шрифты
\usepackage{euscript}	 % Шрифт Евклид
\usepackage{mathrsfs} % Красивый матшрифт

%% Свои команды
\DeclareMathOperator{\sgn}{\mathop{sgn}}

%\setlength\parindent{0ex}
%\setlength\parskip{0.3cm}

%%% Заголовок
\author{Волков Павел А-14-19}
\title{Типовой расчет №6 по численным методам Вариант 3}
\date{\today}

\usepackage{graphicx}

\begin{document} % конец преамбулы, начало документа

\maketitle

\newpage
\section*{Задание}
Записать $LU$ разложение матрицы $A$ из задачи 5 (не проводя дополнительных расчетов). Используя полученное разложение, найти решение системы $Ax = d$
\[
	A = 
	\begin{pmatrix}
		-5 & 1 & 1 & 0 \\
		15 & -6 & 5 & 1 \\
		50 & -16 & 11 & 9 \\
		0 & 6 & 24 & 44
	\end{pmatrix}, d = 
	\begin {pmatrix}
		-61 \\ 122 \\ 373 \\ -698
	\end{pmatrix}
\]

\section*{Решение}

Запишем $LU$ разложение:
\begin{gather*}
	L = 
	\begin{pmatrix}
		1 & 0 & 0 & 0 \\
		-3 & 1 & 0 & 0 \\
		-10 & 2 & 1 & 0 \\
		0 & -2 & 8 & 1
	\end{pmatrix},
	U = 
	\begin{pmatrix}
		-5 & 1 & 1 & 0 \\
		0 & -3 & 8 & 1 \\
		0 & 0 & 5 & 7 \\
		0 & 0 & 0 & -10
	\end{pmatrix}
\end{gather*}
 
Теперь решим систему $Ly = d$:

\begin{gather*}
	\begin{pmatrix}
		1 & 0 & 0 & 0 & -61\\
		-3 & 1 & 0 & 0 & 122\\
		-10 & 2 & 1 & 0 & 373\\
		0 & -2 & 8 & 1 & -698
	\end{pmatrix} \sim
	\begin{pmatrix}
		1 & 0 & 0 & 0 & -61\\
		0 & 1 & 0 & 0 & -61\\
		-10 & 2 & 1 & 0 & 373\\
		0 & -2 & 8 & 1 & -698
	\end{pmatrix} \sim \\ \sim
	\begin{pmatrix}
		1 & 0 & 0 & 0 & -61\\
		0 & 1 & 0 & 0 & -61\\
		0 & 0 & 1 & 0 & -115\\
		0 & 0 & 0 & 1 & 100
	\end{pmatrix}, y = 
	\begin {pmatrix}
		-61 \\ -61 \\ -115 \\ 100
	\end{pmatrix}
\end{gather*}

Осталось решить уравнение $Ux = y$
\begin{gather*}
	\begin{pmatrix}
		-5 & 1 & 1 & 0 & -61\\
		0 & -3 & 8 & 1 & -61\\
		0 & 0 & 5 & 7 & -115\\
		0 & 0 & 0 & -10 & 100
	\end{pmatrix} \sim 
	\begin{pmatrix}
		-5 & 1 & 1 & 0 & -61\\
		0 & -3 & 8 & 1 & -61\\
		0 & 0 & 1 & 0 & -9\\
		0 & 0 & 0 & 1 & -10
	\end{pmatrix} \sim \\ \sim
	\begin{pmatrix}
		1 & 0 & 0 & 0 & 9\\
		0 & 1 & 0 & 0 & -7\\
		0 & 0 & 1 & 0 & -9\\
		0 & 0 & 0 & 1 & -10
	\end{pmatrix}
\end{gather*}

Ответ: $(9, -7, -9, -10)^T$

\end{document}