\documentclass[a4paper,12pt]{report} % добавить leqno в [] для нумерации слева

%%% Работа с русским языком
\usepackage{cmap}					% поиск в PDF
\usepackage{mathtext} 				% русские буквы в формулах
\usepackage[T2A]{fontenc}			% кодировка
\usepackage[utf8]{inputenc}			% кодировка исходного текста
\usepackage[english,russian]{babel}	% локализация и переносы

%%% Дополнительная работа с математикой
\usepackage{amsmath,amsfonts,amssymb,amsthm,mathtools} % AMS
\usepackage{icomma} % "Умная" запятая: $0,2$ --- число, $0, 2$ --- перечисление

%% Номера формул
\mathtoolsset{showonlyrefs=true} % Показывать номера только у тех формул, на которые есть \eqref{} в тексте.

%% Шрифты
\usepackage{euscript}	 % Шрифт Евклид
\usepackage{mathrsfs} % Красивый матшрифт

%% Свои команды
\DeclareMathOperator{\sgn}{\mathop{sgn}}

%\setlength\parindent{0ex}
%\setlength\parskip{0.3cm}

%%% Заголовок
\author{Волков Павел А-14-19}
\title{Типовой расчет №7 по численным методам Вариант 3}
\date{\today}

\usepackage{graphicx}

\begin{document} % конец преамбулы, начало документа

\maketitle

\newpage
\section*{Задание}
Решить систему уравнений $Ax = b$ методом Холецкого
\[
	A = 
	\begin{pmatrix}
		16 & 36 & 8\\
		36 & 82 & 25\\
		8 & 25 & 102
	\end{pmatrix}, b = 
	\begin {pmatrix}
		-108 \\ -297 \\ -775
	\end{pmatrix}
\]

\section*{Решение}

Запишем общий вид $LU$ разложения в методе Холецкого:

\begin{gather*}
	LL^T = 
	\begin{pmatrix}
		l_{1,1} & 0 & 0 \\
		l_{2,1} & l_{2,2} & 0 \\
		l_{3,1} & l_{3,2} & l_{3,3}
	\end{pmatrix}
		\begin{pmatrix}
		l_{1,1} & l_{2,1} & l_{3,1} \\
		0  & l_{2,2} & l_{3,2}\\
		0 & 0 & l_{3,3}
	\end{pmatrix} = 
	\begin{pmatrix}
		16 & 36 & 8\\
		36 & 82 & 25\\
		8 & 25 & 102
	\end{pmatrix} \\
\end{gather*}

Первая строка:
\begin{gather*}
	l_{1,1}^2 = 16, l_{1,1} = 4 \\
	l_{1,1} \cdot l_{2,1} = 36, l_{2,1} = 9 \\
	l_{3,1} \cdot l_{1,1} = 8, l_{3,1} = 2
\end{gather*}

Вторая строка:
\begin{gather*}
	l_{1,1} \cdot l_{2,1} = 36 \\
	l_{2,1}^2 + l_{2,2}^2 = 82, l_{2,2} = 1 \\
	l_{3,1} \cdot l_{2,1} + l_{3,2} \cdot l_{2,2} = 25, l_{3,2} = 7
\end{gather*}

Третья строка:
\begin{gather*}
	l_{1,1} \cdot l_{3,1} = 8 \\
	l_{2,1} \cdot l_{3,1} + l_{2,2} \cdot l_{3,2} = 25 \\
	l_{3,1}^2 + l_{3,2}^2 + l_{3,3}^2 = 102, l_{3,3} = 7
\end{gather*}

В итоге получили следующее разложение:

\begin{gather*}
	L = 
	\begin{pmatrix}
		4 & 0 & 0 \\
		9 & 1 & 0 \\
		2 & 7 & 7
	\end{pmatrix}
\end{gather*}

Решим систему $Ly = b$

\begin{gather*}
	\begin{pmatrix}
		4 & 0 & 0 & -108 \\
		9 & 1 & 0 & -297 \\
		2 & 7 & 7 & -775
	\end{pmatrix} \sim 
	\begin{pmatrix}
		1 & 0 & 0 & -27 \\
		0 & 1 & 0 & -54 \\
		0 & 0 & 1 & -49
	\end{pmatrix}
\end{gather*}

Заключительный этап решения - $L^Tx = y$

\[
	\begin{pmatrix}
		4 & 9 & 2 & -27 \\
		0 & 1 & 7 & -54 \\
		0 & 0 & 7 & -49
	\end{pmatrix} \sim
	\begin{pmatrix}
		1 & 0 & 0 & 8 \\
		0 & 1 & 0 & -5 \\
		0 & 0 & 1 & -7
	\end{pmatrix}
\]

Ответ: $(8, -5, -7)^T$

\end{document}