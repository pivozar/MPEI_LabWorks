\documentclass[a4paper,12pt]{report} % добавить leqno в [] для нумерации слева

%%% Работа с русским языком
\usepackage{cmap}					% поиск в PDF
\usepackage{mathtext} 				% русские буквы в формулах
\usepackage[T2A]{fontenc}			% кодировка
\usepackage[utf8]{inputenc}			% кодировка исходного текста
\usepackage[english,russian]{babel}	% локализация и переносы

\usepackage{graphicx}				%вставка изображений(графиков, в частности)
\usepackage{listings}
\usepackage{color}

\definecolor{dkgreen}{rgb}{0,0.6,0}
\definecolor{gray}{rgb}{0.5,0.5,0.5}
\definecolor{mauve}{rgb}{0.58,0,0.82}

\lstset{frame=tb,
  language=Python,
  aboveskip=3mm,
  belowskip=3mm,
  showstringspaces=false,
  columns=flexible,
  basicstyle={\small\ttfamily},
  numbers=none,
  numberstyle=\tiny\color{gray},
  keywordstyle=\color{blue},
  commentstyle=\color{dkgreen},
  stringstyle=\color{mauve},
  breaklines=true,
  breakatwhitespace=true,
  tabsize=4
}

%%% Дополнительная работа с математикой
\usepackage{amsmath,amsfonts,amssymb,amsthm,mathtools} % AMS
\usepackage{icomma} % "Умная" запятая: $0,2$ --- число, $0, 2$ --- перечисление

%% Номера формул
\mathtoolsset{showonlyrefs=true} % Показывать номера только у тех формул, на которые есть \eqref{} в тексте.

%% Шрифты
\usepackage{euscript}	 % Шрифт Евклид
\usepackage{mathrsfs} % Красивый матшрифт

%% Свои команды
\DeclareMathOperator{\sgn}{\mathop{sgn}}

\usepackage{parskip}
%\setlength\parindent{0ex}
%\setlength\parskip{0.3cm}

\renewcommand{\arraystretch}{1.8}

%%% Заголовок
\author{Волков Павел А-14-19}
\title{Отчет по Лабораторной работе №5}
\date{\today}

\begin{document}

\begin{titlepage}
	\newpage

	\begin{center}
	НАЦИОНАЛЬНЫЙ ИССЛЕДОВАТЕЛЬСКИЙ УНИВЕРСИТЕТ\\
		"МОСКОВСКИЙ ЭНЕРГЕТИЧЕСКИЙ ИНСТИТУТ"\\
	\end{center}

	\vspace{8em}	

	\begin{center}
		\Large Кафедра математического и компьютерного моделирования\\ 
	\end{center}

	\vspace{2em}

	\begin{center}
		\textsc{\textbf{ \Large Численные методы \linebreak Отчет по лабораторной работе №5 \linebreak "Численное интегрирование." \linebreak Вариант 52}}
	\end{center}

	\vspace{6em}



	\newbox{\lbox}
	\savebox{\lbox}{\hbox{Амосова Ольга Алексеевна}}
	\newlength{\maxl}
	\setlength{\maxl}{\wd\lbox}
	\hfill\parbox{11cm}{
		\hspace*{5cm}\hspace*{-5cm}Студент:\hfill\hbox to\maxl{Волков Павел Евгеньевич\hfill}\\
		\hspace*{5cm}\hspace*{-5cm}Преподаватель:\hfill\hbox to\maxl{Амосова Ольга Алексеевна}\\
		\\
		\hspace*{5cm}\hspace*{-5cm}Группа:\hfill\hbox to\maxl{А-14-19}\\
	}


	\vspace{\fill}

	\begin{center}
		Москва \\2021
	\end{center}

\end{titlepage}
\section*{Задача 5.1}
\subsection*{Постановка задачи}
Вычислить значение интеграла $I = \int\limits_1^3P_m(x)dx$, где $P_m(x) = \sum\limits_{i=0}^mc_ix^i$, с помощью квадратурных формул левых прямоугольников, Гаусса и по формуле индивидуального варианта(правило 3/8).
\[
\begin{array}{| c | c | c | c | c | c | c |}
	\hline
	   №    & c_0 & c_1 & c_2 & c_3 & c_4 & c_5 \\ \hline
	5.1.52 &  5.4 & 2.1 & 0.3 &  2.1 & 1.6 & 1.4 \\ \hline
\end{array}
\]
\subsection*{Решение}
 Запишем формулу левых прямоугольников:
\begin{gather*}
	S = h\sum\limits_{i=0}^{n-1}f(x_i) \\
	R = \dfrac{M_1(b-a)}{2}h
\end{gather*}

Формула правила 3/8:
\begin{gather*}
	S = \dfrac{h}{8}\sum\limits_{i=1}^n\left[f(x_{i-1}) + 3f\left(x_{i-1} + \dfrac{h}{3}\right) + 3f\left(x_i - \dfrac{h}{3}\right) + f(x_i)\right]\\
	R = \dfrac{M_4(b-a)}{6480}h^4
\end{gather*}

Так как формула Гаусса точна для многочленов степени $2N + 1$ при $N + 1$ узлах, а степень исходного многочлена 5, то для вычисления интеграла без погрешности достаточно взять 3 узла:
\[
	\int\limits_1^3f(t)dt = A_0f(t_0) + A_1f(t_1) + A_2f(t_2)
\]
Так как у нас 6 неизвестных, то следует взять 6 первых базисных функций: $1, t, t^2, t^3, t^4, t^5$. Будем строить формулу для отрезка $[-1, 1]$ а затем выполним линейное преобразование.
\begin{gather*}
	\int\limits_{-1}^11dt = 2 =A_0 + A_1 + A_2 \\
	\int\limits_{-1}^1tdt = 0 =A_0t_0 + A_1t_1 + A_2t_2 \\
	\int\limits_{-1}^1t^2dt = 2/3 = A_0t_0^2 + A_1t_1^2 + A_2t_2^2 \\
	\int\limits_{-1}^1t^3dt = 0 = A_0t_0^3 + A_1t_1^3 + A_2t_2^3 \\
	\int\limits_{-1}^1t^4dt = 2/5 = A_0t_0^4 + A_1t_1^4 + A_2t_2^4 \\
	\int\limits_{-1}^1t^5dt = 0 = A_0t_0^5 + A_1t_1^5 + A_2t_2^5
\end{gather*}

Из получившейся системы имеем такие решения: $A_0 = A_2 = 5/9$, $A_1 = 8/9$, $t_1 = 0$, $t_0 = -t_2 = -\sqrt{3/5}$

Окончательно, получили квадратурную формулу Гаусса с 3-мя узлами:
\[
	\int\limits_{-1}^1f(t)dt = \dfrac{5}{9}f\left( -\sqrt{\dfrac{3}{5}}\right) + \dfrac{8}{9}f(0) + \dfrac{5}{9}f\left(\sqrt{\dfrac{3}{5}}\right)
\]

После линейной замены формула примет вид:
\[
	\int\limits_{1}^3f(t)dt = \dfrac{5}{9}f\left(2 -\sqrt{\dfrac{3}{5}}\right) + \dfrac{8}{9}f(2) + \dfrac{5}{9}f\left(2+\sqrt{\dfrac{3}{5}}\right)
\]

Таблица 5.1(Интеграл вычислялся с точностью $\varepsilon = 0.01$)
\[
\begin{tabular}{| p{5cm} | p{5cm} | p{5cm} |}
	\hline
	 Найденное точное значение интеграла \newline$I = 311.1066666666666$ & Число разбиений отрезка $n$ \newline Шаг интегрирования $h$ & Значение интеграла, вычисленное по составной формуле $I^h$ \newline Величина погрешности интеграла, вычисленного по составной формуле $R^h$ \\ \hline
	Метод левых прямоугольников &  $n = 160080$ \newline $h = 0.0000124937$ & $I^h = 311.10336832596073$ \newline $R^h = 0.01$ \\ \hline
	Правило 3/8 &  $n = 5$ \newline $h = 0.4$ & $I^h = 311.1096248888889$ \newline $R^h = 0.004$ \\ \hline
	Метод Гаусса &  Число узлов квадратуры $N = 3$ & $I^G = 311.1066666666667$ \\ \hline
\end{tabular}
\]
\section*{Задача 5.2}
\subsection*{Постановка задачи}
Вычислить интеграл $ \int\limits_a^bf(x)dx $ с точностью $\varepsilon = 10^{-12}$.
\[
\begin{array}{| c | c | c |}
	\hline
	№ & f(x) & [a, b] \\ \hline
	5.2.52 & 6e^{-x}\sin{2\pi x} & [0, 3] \\ \hline
\end{array}
\]
\subsection*{Решение}
Для вычисления интеграла будем использовать правило 3/8. Запишем формулу оценки погрешности по правилу Рунге:
\[
	R^h = \dfrac{I^h - I^{2h}}{2^p - 1}
\]

Приведем таблицу результатов для вычисления интеграла с точностью $\varepsilon < 10^{-12}$:

\begin{tabular}{| p{7.5cm} | p{7.5cm} |}
	\hline
	$I = 0.8849699594107223$ & Правило 3/8 \\ \hline
	Число разбиений отрезка	& $n = 2048$ \\ \hline
	Значение интеграла & $I^h = 0.8849699594116519$ \\ \hline
	Погрешность & $R^h = 0.9295897385186436 \cdot 10^{-12}$ \\ \hline
	Уточненное значение по правилу Рунге & $I = 0.8849699594107236$\\ \hline
	Погрешность уточненного значения & $R = 1.3322676295501878 \cdot 10^{-15}$\\ \hline
\end{tabular}

\end{document}