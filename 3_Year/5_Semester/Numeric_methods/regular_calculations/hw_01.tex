\documentclass[a4paper,12pt]{report} % добавить leqno в [] для нумерации слева

%%% Работа с русским языком
\usepackage{cmap}					% поиск в PDF
\usepackage{mathtext} 				% русские буквы в формулах
\usepackage[T2A]{fontenc}			% кодировка
\usepackage[utf8]{inputenc}			% кодировка исходного текста
\usepackage[english,russian]{babel}	% локализация и переносы

%%% Дополнительная работа с математикой
\usepackage{amsmath,amsfonts,amssymb,amsthm,mathtools} % AMS
\usepackage{icomma} % "Умная" запятая: $0,2$ --- число, $0, 2$ --- перечисление

%% Номера формул
\mathtoolsset{showonlyrefs=true} % Показывать номера только у тех формул, на которые есть \eqref{} в тексте.

%% Шрифты
\usepackage{euscript}	 % Шрифт Евклид
\usepackage{mathrsfs} % Красивый матшрифт

%% Свои команды
\DeclareMathOperator{\sgn}{\mathop{sgn}}

%\setlength\parindent{0ex}
%\setlength\parskip{0.3cm}

%%% Заголовок
\author{Волков Павел А-14-19}
\title{Типовой расчет №21 по численным методам Вариант 3}
\date{\today}

\usepackage{graphicx}

\begin{document} % конец преамбулы, начало документа

\section*{Задание №28}

Используя понятие точности на многочленах, постройте квадратурную формулу следующего вида и укажите ее алгебраический порядок точности.
\[
	а) \int\limits_0^2f(x)dx \approx Af(0) + Bf(1)
\]

\section*{Решение}
Будем использовать метод неопределенных коэфициентов для построения квадратурной формулы. В качестве базиса возьмем функции $1, x$

\begin{gather*}
	 \int\limits_0^21dx = 2 = A + B \\
	 \int\limits_0^2xdx = 2 = B \\
\end{gather*}

В итоге получили формулу $\int\limits_0^2f(x)dx \approx 2f(1)$. Заметим, что получилась формула центральных прямоугольников, если принять  шаг $h=2$, следовательно, и формула будет иметь второй порядок точности по $h$.

\end{document}