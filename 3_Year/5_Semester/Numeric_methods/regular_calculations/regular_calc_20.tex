\documentclass[a4paper,12pt]{report} % добавить leqno в [] для нумерации слева

%%% Работа с русским языком
\usepackage{cmap}					% поиск в PDF
\usepackage{mathtext} 				% русские буквы в формулах
\usepackage[T2A]{fontenc}			% кодировка
\usepackage[utf8]{inputenc}			% кодировка исходного текста
\usepackage[english,russian]{babel}	% локализация и переносы

%%% Дополнительная работа с математикой
\usepackage{amsmath,amsfonts,amssymb,amsthm,mathtools} % AMS
\usepackage{icomma} % "Умная" запятая: $0,2$ --- число, $0, 2$ --- перечисление

%% Номера формул
\mathtoolsset{showonlyrefs=true} % Показывать номера только у тех формул, на которые есть \eqref{} в тексте.

%% Шрифты
\usepackage{euscript}	 % Шрифт Евклид
\usepackage{mathrsfs} % Красивый матшрифт

%% Свои команды
\DeclareMathOperator{\sgn}{\mathop{sgn}}

%\setlength\parindent{0ex}
%\setlength\parskip{0.3cm}

%%% Заголовок
\author{Волков Павел А-14-19}
\title{Типовой расчет №20 по численным методам Вариант 3}
\date{\today}

\usepackage{graphicx}

\begin{document} % конец преамбулы, начало документа

\maketitle

\newpage
\section*{Задание}
Вычислить приближенное значение интеграла $\int\limits^a_b f(x)dx$, используя квадратурные формулы: а) центральных прямоугольников с шагом $h = 0.4$; дать априорную оценку погрешности; б) трапеций с шагами $h = 0.4$ и $h = 0.2$; оценить погрешность последнего результата по правилу Рунге и уточнить последний рузультат по  Рунге; в) Симпсона $h = 0.4$

УКАЗАНИЕ. Промежуточные результаты вычислять с шесьтью знаками после запятой. Аргументы тригонометрических функций вычислять в радианах

\[
	f(x) = e^{\sin^2{x}}, a = 0.8, b = 2.4
\]

\section*{Решение}

\subsection*{Формула центральных прямоугольников}
Запишем формулу центральных прямоугольников: $h\sum\limits_{i=1}^{n}f_{i-1/2} = 0.4(2.030076 + 2.640877 + 2.581522 + 1.922576) = 0.4 \times 9.175052= 3.670021$

Априорная оценка погрешности: $R \leq \dfrac{M_2(b-a)h^2}{24} = \dfrac{1.842235 \times 1.6 \times 0.16}{24} = 0.0196505$

\subsection*{Формула трапеций}
С шагом $h = 0.4$
\begin{multline}
	I^{трап} = h(\dfrac{f_0 + f_n}{2} + \sum\limits_{i=1}^{n-1}f_i) = 
	0.4 \times (\dfrac{1.672968 + 1.578145}{2} \\
	+ 2.383802 + 2.715965 + 2.286041 ) = 3.604546
\end{multline}

\noindentС шагом $h = 0.2$
\begin{multline}
	I^{трап} = h(\dfrac{f_0 + f_n}{2} + \sum\limits_{i=1}^{n-1}f_i) = 0.2 \times (\dfrac{1.672968 + 1.578145}{2} +\\ 		2.030076 + 2.383802 +2.640877 + 2.715965 + 2.581522 + 2.286041 + 1.922576 ) = 3.637283
\end{multline}
Оценка погрешности по Рунге: $I - I^{h=0.2} \approx \dfrac{I^h - I^{2h}}{2^p - 1} = \dfrac{3.637283 - 3.604546}{2^2 -  1} = 0.010912$

Тогда после уточнения второго результата по Рунге получаем следующее значение интеграла: $I \approx I^{h=0.2} + \dfrac{I^h - I^{2h}}{2^p - 1} =  3.637283 + 0.010912 =3.648195$

\subsection*{Формула Симпсона}
\begin{multline}
I^{Симпсона} = \dfrac{h}{6} (f_0 + f_n + 2\sum\limits_{i=1}^{n-1}f_i +  4\sum\limits_{i=1}^{n}f_{f_{i-1/2}} ) = \dfrac{0.4}{6} (1.672968 + 1.578145 + 2 \times7.385809 + \\ 4 \times9.175052 ) = 3.648196
\end{multline}

\end{document}