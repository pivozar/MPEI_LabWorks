\documentclass[a4paper,12pt]{report} % добавить leqno в [] для нумерации слева

%%% Работа с русским языком
\usepackage{cmap}					% поиск в PDF
\usepackage{mathtext} 				% русские буквы в формулах
\usepackage[T2A]{fontenc}			% кодировка
\usepackage[utf8]{inputenc}			% кодировка исходного текста
\usepackage[english,russian]{babel}	% локализация и переносы

%%% Дополнительная работа с математикой
\usepackage{amsmath,amsfonts,amssymb,amsthm,mathtools} % AMS
\usepackage{icomma} % "Умная" запятая: $0,2$ --- число, $0, 2$ --- перечисление

%% Номера формул
\mathtoolsset{showonlyrefs=true} % Показывать номера только у тех формул, на которые есть \eqref{} в тексте.

%% Шрифты
\usepackage{euscript}	 % Шрифт Евклид
\usepackage{mathrsfs} % Красивый матшрифт

%% Свои команды
\DeclareMathOperator{\sgn}{\mathop{sgn}}

%\setlength\parindent{0ex}
%\setlength\parskip{0.3cm}

%%% Заголовок
\author{Волков Павел А-14-19}
\title{Типовой расчет №21 по численным методам Вариант 3}
\date{\today}

\usepackage{graphicx}

\begin{document} % конец преамбулы, начало документа

\maketitle

\newpage
\section*{Задание}
Дан интеграл вида: $\int\limits_a^b(c_0 + c_1x + c_2x^2 + c_3x^3 + c_4x^4)dx$. Используя априорную оценку погрешности формулы трапеций, определить шаг интегрирования, достаточный для достижения точности $\varepsilon = 0.01$, и вычислить интеграл с этим шагом. Вычислив точное значение интеграла, подтвердить достижение указанной точности.
\[
\begin{array}{ | c | c | c | c | c | c | c | }
	\hline
	    a  &   b  & c_0 & c_1 & c_2 & c_3 & c_4  \\ \hline
	 -0.4 & 0.1 &  -3  &   2  &   2  &   0   &    2  \\ \hline
\end{array}
\]
\section*{Решение}
Запишем априорную оценку формулы трапеций: $R \leq \dfrac{M_2(b-a)h^2}{12}$
\begin{gather*}
	\varepsilon \geq \dfrac{M_2h^2}{24} \\
	0.01 \geq 5.926666 \times h^2 \\
	0.1 \geq 2.434474 \times h \\
	0.041076 \geq h
\end{gather*}

Найдем число разбиений отрезка $[a, b]$: $n = (b-a) / h =\lceil12.172373\rceil = 13 $ и пересчитаем шаг: $h_2 = 0.5 / 13 = 0.038461$

Вычисляем значение интеграла по формуле трапеций:
\begin{gather*}
	I^h = h\left(\dfrac{P(a) + P(b)}{2} + \sum\limits_{i=1}^{13}(P(a + ih)\right) = -1.6022561 \\
	|I - I^h| = |-1.6022566 - (-1.6022561)| = 0.0000005 < \varepsilon
\end{gather*}


\end{document}